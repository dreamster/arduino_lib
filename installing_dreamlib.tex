\documentclass[12pt]{article}
\usepackage{fancyhdr}
\usepackage{hyperref}
\usepackage{titling}
\newcommand{\subtitle}[1]{%
  \posttitle{%
    \par\end{center}
    \begin{center}\large#1\end{center}
    \vskip0.5em}%
}

\pagestyle{fancy}

%Title for the document
\title{How to include Dreamlib on my arduino IDE}

%Version of the document
\subtitle{Version 0.1.1}

%Include current date on the file
\date{\today}

% For the headers and footers, use Xhead and Xfoot, were X can be {l,c,r}
% example: \rfoot{this goes on the footer, right side}

\begin{document}
\maketitle

First of all, you will need to download the \textit{dreamlib.zip} file, 
located on our \href{https://github.com/dreamster/arduino_lib}
{Github project page}. To do this, click on \textit{download zip} button, 
located on the bottom left side of the site, or clone the repository, which 
will also include the file.

If the version of your IDE is 1.0.5, the installation of Dreamlib can be done 
as follows:

In the Arduino IDE, navigate to Sketch $\rightarrow$ Import Library. At the 
top of the drop down list, select the option to \textit{Add Library}. 

You will be prompted to select the library you would like to add. Navigate to 
the \textit{Dreamlib.zip} file's location and open it (this step requires the 
file to be zipped, as downloaded).

Return to the Sketch $\rightarrow$ Import Library menu. You should now see the 
Dreamlib at the bottom of the drop-down menu. It is ready to be used in your 
sketch, although you wont be able to see the examples until after you restart 
your IDE.

\begin{thebibliography}{1}

\bibitem{Oficial Site Arduino}
  Arduino tutorials,
  \emph{\href{http://arduino.cc/en/Guide/Libraries}{oficial page}}.

	
\bibitem{Oficial Site dreamster}
  Dreamster Github,
  \emph{\href{https://github.com/dreamster}{main directory}}.

\end{thebibliography}

\end{document}

%Done using dreamsters basic template
